\documentclass[11pt,a4paper]{article}
\usepackage{geometry}
\geometry{
    a4paper,
    total={170mm,257mm},
    left=20mm,
    top=20mm,
}
\usepackage{color}
\PassOptionsToPackage{hyphens}{url}\usepackage{hyperref}
\usepackage{amsmath}
\usepackage{amssymb}
\usepackage[ruled,vlined]{algorithm2e}
\usepackage{xargs}                      % Use more than one optional parameter in a new commands
\usepackage{xcolor}  % Coloured text etc.

\newcommandx{\syed}[2]{{\uline{#1}}\textcolor{orange}{(#2)}}

\setlength\parindent{0pt}
\newcommand{\todo}[1]{\textcolor{red}{TODO: #1}}
\newcommand{\SubItem}[1]{
    {\setlength\itemindent{15pt} \item[-] #1}
}

\newcommand{\TWF}{\ensuremath{\mathcal{W}}}

\begin{document}
\title{Polkadot Weights}
\author{Web3 Foundation}
\date{May 2020}
\maketitle

\section{Motivation}
Polkadot network like any other permissionless system need to implement a mechanism to measure and to limit usage in order to prevent the network overload and to mitigate DoS vulnerabilities. In particular, Polkadot enforces a limited time-window for block producers to create a block, including limitations on block size which can make the selection and execution
of certain extrinsics too expensive and decelerate the network.

In contrast to some other systems such a Ethereum which implement fine measurement of each low level operation know as gas metering, Polkadot takes a more relaxed approach by implementing a measuring system of coarser granularity called transaction weight system.

The weight
system introduces a mechanism for block producers to measure the expense of
extrinsics and determine how "heavy" it is. With this mechanism, block producers
can select a set of extrinsics and saturate the block to it's fullest potential
without exceeding any limitations (as described in section \ref{sec:limitations}).
\newline

Additionally, Polkadot also introduces a specified block ratio (as defined in section \ref{sec:limitations}),
ensuring that only a certain portion of the total block size gets used for regular extrinsics.
The remaining space is reserved for critical, operational extrinsics required for the functionality
by Polkadot itself.

We start this document with introducing the assumption upon which the polkadot transaction weight system is designed in Section \ref{sec:assumptions}. In Section \ref{sec:limitation} We discuss the limitation Polkadot needs to enforce on Block size. In Section \ref{sec:runtime-primitives}, we describe in detail the procedure upon which the weight of any transaction should be calculated. In Section \ref{sec:practical-examples}, we present how we apply this procedure to compute the weight of particular runtime functions.

\section{Assumptions}\label{sec:assumption}

In this section we define the concept of weight and we discuss the considerations need to be accounted for when assigning weight to transactions. These considerations are essentials in order for the weight system to delivering its fundamental mission, i.e. fair distribution of network resource and preventing a network overload, listed as follows:

\begin{itemize}
\item to prevent the block being filled with too many extrinsics
\item to avoid extrinsics where its execution takes too long, by assigning a transaction fees to each extrinsic proportional to their resource consumption.
\end{itemize}

\begin{definition}
  \label{defn:weight-function}
  The {\b Polkadot transaction weight function} denoted by $\mathcal{W}$ as follows:
  \[
  \mathcal{W}: \mathcal{E}\rightarrow \mathbb{N}
  \mathcal{W}: E \mapsto w
  \]

  where $w$ is a positive integer representing the weight of the extrinsic $E$.
  
\end{definition}

In the remainder of this section we discuss the requirements to which the weight function needs to comply to.

\begin{itemize}

\item Computations of function $\mathcal{W}(E)$  must be determined before execution of that $E$.
\item Due to the limited time window, computations of $\TWF$ must be done quickly and consume
      few resources themselves.
\item $\TWF$ must be self contained and must not require I/O on the chain state. $\TWF(E)$
must depends solely on the Runtime function representing $E$ and its parameters.
  
\end{itemize}

\section{Limitations}\label{sec:limitations}
The assigned weights should be relative to each others execution time and "heaviness",
although weights can be assigned depending on the priorities the chain is supposed to endorse.
Following limitations must be considered when assigning weights, which vary on the Runtime.

\subsection{Considerable limitations}
\begin{itemize}
\item Maximum block length
\item Maximum block weight
\item Targeted time per block
\item Available block ration reserved for normal, none-operational transactions
\end{itemize}

\subsection{Considerable limitations in Polkadot}
As of the official Polkadot Runtime, the limitations are set as follows:

\begin{itemize}
\item Maximum block length: $5 \times 1'024 \times 1'024 = 5'242'880$ Bytes
\item Maximum block weight: 2'000'000'000'000
\item Targeted time per block: 6 seconds
\item Available block ratio: 75\%
\end{itemize}

The values of the assigned weight itself is not relevant. It must only fulfill the requirements
as noted by the fundamentals and limitations, and can be assigned as the author sees fit.
As a simple example, consider a maximum block weight of 1'000'000'000, an available ratio of
75\% and a targeted transaction throughput of 500 transactions, we could assign the weight
for each transaction at about 1'500'000.
\newline

Do note that the smallest, non-zero weight in Polkadot is set at 10'000.

\section{Runtime Primitives}
\label{sec:runtime-primitives}
The Runtime functions must be studied in order to determine which parts of the code will excessively
increase execution time. Potential indicators like loops, IO operations and data manipulation must
be considered. Based on those observations, Weights should be applied on operations
that could have heavy implications on those design choices. The final assigned Weights are calculated
by benchmarking the Runtime functions, which is described in section \ref{sect:benchmarking}.
\newline

Section \ref{sect:examples-runtime-analysis} walks through two practical examples of Runtime function
analysis.
Not every possible outcome can be caused with input parameters to the Runtime function.
In some circumstances, preliminary work regarding
storage is required before a specific benchmark can be reliably measured.
Two of such examples are described in section \ref{sect:examples-preliminary-work}.

\subsection{Primitive Types}\label{sect:primitive-types}
The Runtime reuses components/primitives to interact with the state storage. The behavior and
the execution cost of those primitives can be studied and a Weight should be applied for each use
within the Runtime code.
\newline

For storage, Polkadot uses three different types of storage types across its modules, depending on the
context:

\begin{itemize}
  \item \textbf{Value}: Operations on a single value.
  \newline\newline
  The final key-value pair is stored under the key:\newline
  \verb|hash(module_prefix) + hash(storage_prefix)|.
  \item \textbf{Map}: Operations on mulitple values, datasets, where each entry has its
  corresponding, unique key.
  \newline\newline
  The final key-value pair is stored under the key:\newline
  \verb|hash(module_prefix) + hash(storage_prefix) + hash(encode(key))|.
  \newpage
  \item \textbf{Double map}: Just like \textbf{Map}, but uses two keys instead of one 
  (child storages).
  \newline\newline
  The final key-value pair is stored under the key:\newline
  \begin{verbatim}
  hash(module_prefix) + hash(storage_prefix)
    + hash(encode(key1)) + hash(encode(key2))
  \end{verbatim}
\end{itemize}

Which type to use depends on the functionality of the Runtime module (or its sub-processes, rather).
In some cases only a single value is required. In others, multiple values need to be operated on.
\newline

Those lower level types get abstracted over in each individual Runtime module using the \verb|decl_storage!|
macro. Therefore, each module specifies its own types that are used as input and output values.
The abstractions do give indicators on what operations must be closely observed and where potential
performance penalties and attack vectors are possible.

\subsubsection{Considerations}\label{sect:primitive-types-considerations}

The storage layout is mostly the same for every primitive type, primarily differentiated
by using special prefixes for the storage key. Big differences arise on how the primitive
types are used in the Runtime function, on whether single values are operated on or entire
datasets. Single value operations are generally quite cheap and its execution time does
not vary depending on the data that's being processed. However, excessive IO overhead can appear
when IO operations are executed repeatedly, such as loops. Especially, when the amount of loop
iterations can be influenced by the caller of the function or by certain conditions in the
state storage.
\newline

Maps, in contrast, have additional overhead when inserting or retrieving
datasets, which vary in sizes. Additionally, the Runtime function has to process each item inside
that list.
\newline

Indicators for performance penalties:

\begin{itemize}
  \item \textbf{Fixed iterations and datasets} -
  Fixed iterations and datasets can increase the overall cost of the Runtime functions,
  but the execution time
  does not vary depending on the input parameters or storage entries.
  A base Weight is appropriate in this case.
  \item \textbf{Adjustable iterations and datasets} -
  If the amount of iterations or datasets depend on the input parameters of the caller or
  specific entries in storage,
  then a certain Weight should be applied for each (additional) iteration or item.
  The Runtime defines the maximum value for such cases. If it doesn't, it unconditionally has to
  and the Runtime module must be adjusted.
  \newline\newline
  When selecting parameters
  for benchmarking, the benchmarks should range from the minimum value to the maximum value, as described
  in paragraph \ref{para:max-value}.
  \item \textbf{Input parameters} -
  Input parameters that users pass on to the Runtime function can result in expensive operations.
  Depending on the data type, it can be appropriate to add additional Weights based on certain properties,
  such as data size, assuming the data type allows varying sizes.
  The Runtime must define limits on those properties. If it doesn't, it unconditionally
  has to and the Runtime module must be adjusted.
  \newline\newline
  When selecting parameters for benchmarking, the benchmarks should range from the minimum values to the
  maximum value, as described in paragraph \ref{para:max-value}.
\end{itemize}

\label{para:max-value} What the maximum value should be really depends on the functionality that the Runtime function
is trying to provide. If the choice for that value is not obvious, then it's advised to run
benchmarks on a big range of values and pick a conservative value below the \verb|targeted time per block|
limit as described in section \ref{sec:limitations}.

\section{Benchmarking}\label{sect:benchmarking}
Assigning weights based on theoretical performance can be
unreliable and too complex due to imprecision in back-end systems, internal communication
within the Runtime and design choices in the software. Therefore, all available Runtime 
functions, which create and execute extrinsics, have to be benchmarked with a large
collection of input parameters.
\newline

In order to select useful parameters, the Runtime functions have to be analysed to fully
understand which behaviors or conditions can result in expensive execution times,
which is described closer in section \ref{sect:primitive-types}.

\subsection{Parameters}
The inputs parameters highly vary depending on the Runtime function and must therefore
be carefully selected. The benchmarks should use input parameters which will most likely be
used in regular cases, as intended by the authors, but must also consider worst case
scenarios and inputs which might decelerate or heavily impact performance of the function.
The input parameters should be randomized in order to cause various effects in behaviors
on certain values, such as memory relocations and other outcomes that can impact performance.
\newline

It's not possible to benchmark every single value. However, one should select a range
of inputs to benchmark, spanning from the minimum value
to the maximum value which will most likely exceed the expected usage of that function.
This is described in more detail in section \ref{sect:primitive-types-considerations}.
\newline

The benchmarks should run individual executions/iterations within that range, where the chosen parameters
should give insight on the execution time and resource cost. Selecting imprecise parameters
or too extreme ranges might indicate an inaccurate result of the function as it will be used in production.
Therefore, when a range of input parameters gets benchmarked, the result of each individual parameter
should be recorded and ideally visualized. The author should then decide on the most probable average
execution time, basing that decision on the limitations of the Runtime and expected usage of the network.
\newline

Additionally, given the distinction theoretical and practical usage, the author reserves the right
to make adjustments to the input parameters and assigned weights according to observed behavior
of the actual, real-world network.

\subsection{Blockchain State}
The benchmarks should be performed on blockchain states that already polluted and contain a history of
extrinsics and storage changes. Runtime functions that require I/O on structures
such as Tries will therefore produce more realistic results that will reflect the real-world
performance of the Runtime.

\subsection{Environment}
The benchmarks should be executed on clean systems without interference of other processes
or software. Additionally, the benchmarks should be executed on multiple machines with different
system resources, such as CPU performance, CPU cores, RAM and storage speed.

\section{Practical examples}\label{sec:practical-examples}

\subsection{Runtime function analysis}\label{sect:examples-runtime-analysis}

This section walks through Runtime functions available in the Polkadot Runtime to
demonstrate the analysis process as described in section \ref{sect:primitive-types}.

\subsubsection{Practical example \#1}

\subsubsection*{Analysis}

In Polkadot, accounts can save information about themselves onchain, known as the "Identity Info".
This includes information such as display name, legal name, email address and so on. Polkadot selects
a set of registrars which can judge identities and therefore incentivizes a reputation model. The judgement itself
is done offchain. The registrars rating, however, is saved onchain, directly in the corresponding 
Identity Info. It's also note worthy that Identiy Info can contain additional fields, set manually by
the corresponding account holder.
\newline

The function \verb|request_judgement| from the \verb|identity| Pallet allows users to request a judgement
from a specific registrar. Studying this function reveals multiple design choices that can impact performance.
\newline

First, it fetches a list of current registrars from storage and then searches that list for the specified
registrar index.

\begin{verbatim}
let registrars = <Registrars<T>>::get();
let registrar = registrars.get(reg_index as usize).and_then(Option::as_ref)
  .ok_or(Error::<T>::EmptyIndex)?;
\end{verbatim}

Then, it searches for the Identity Info from storage, based on the sender of the transaction.

\begin{verbatim}
let mut id = <IdentityOf<T>>::get(&sender).ok_or(Error::<T>::NoIdentity)?;
\end{verbatim}

The Identity Info contains the entirety of entries, in an ordered form. It then proceeds the search all
those entries for the specified registrar index. If an entry can be found, the value is updated
(assuming the registrar is not "stickied", which implies it cannot be changed). In the context of registrars,
this update implies that the Identity Info should be rejudged. If the entry cannot 
be found, the value is inserted into the index where a matching element could be inserted while maintaining
sorted order. This results in memory reallocation.

\begin{verbatim}
match id.judgements.binary_search_by_key(&reg_index, |x| x.0) {
  Ok(i) => if id.judgements[i].1.is_sticky() {
    Err(Error::<T>::StickyJudgement)?
  } else {
    id.judgements[i] = item
  },
  Err(i) => id.judgements.insert(i, item),
}
\end{verbatim}

After that, it proceeds to reserve the registrar fee, inserts the newly updated identity info into storage
and deposits the event into the scheduler.

\begin{verbatim}
T::Currency::reserve(&sender, registrar.fee)?;
<IdentityOf<T>>::insert(&sender, id);
Self::deposit_event(RawEvent::JudgementRequested(sender, reg_index));
\end{verbatim}

\subsubsection*{Considerations}

Based on the considerations described in section \ref{sect:primitive-types-considerations},
the analysis reveals multiple, critical points. Those must be covered in the benchmarking
processes described in section \ref{sect:benchmarking} and require preliminary work as
described in section \ref{sect:examples-preliminary-work}.
\newline

Key points:
\begin{itemize}
  \item Varying amount of registrars.
  \item Varying amount of preexisting accounts in storage.
  \item The specified registrar is searched for in the Identity Info. Additionally, if a new value gets
  inserted into the byte array, memory get reallocated. Depending on the size of the Identity Info, the
  execution time can vary.
  \item Varying sizes of Identity Info, including additional fields.
  \item Inserting new registrars into Identity Info results in memory reallocation.
  \item It is legitimate to introduce additional Weights for changes the sender has influence over,
  such the additional fields in the Identity Info.
\end{itemize}

\subsubsection{Practical example \#2}

\subsubsection*{Analysis}

The function \verb|payout_stakers| from the \verb|staking| Pallet can be called by a single account in order
to payout the reward for all nominators who back a particular validator. The reward also covers the validator's
share. This function is interesting because it iterators over a range of nominators, which varies, and does
IO operation for each of them.
\newline

First, this function makes some basic checks to verify if the specified era is not higher then the current era
(future) and is within the allowed range ("history depth"), specified by the Runtime.
After that, it fetches the era payout from storage
and additionally verifies whether the specified account is indeed a validator and receives the corresponding
"Ledger".

\begin{verbatim}
let era_payout = <ErasValidatorReward<T>>::get(&era)
  .ok_or_else(|| Error::<T>::InvalidEraToReward)?;

let controller = Self::bonded(&validator_stash).ok_or(Error::<T>::NotStash)?;
let mut ledger = <Ledger<T>>::get(&controller).ok_or_else(|| Error::<T>::NotController)?;
\end{verbatim}

The Ledger keeps a list of tracked rewards. The function only retains the entries of the "history depth",
and conducts a binary search for the specified era.

\begin{verbatim}
ledger.claimed_rewards.retain(|&x| x >= current_era.saturating_sub(history_depth));
match ledger.claimed_rewards.binary_search(&era) {
  Ok(_) => Err(Error::<T>::AlreadyClaimed)?,
  Err(pos) => ledger.claimed_rewards.insert(pos, era),
}
\end{verbatim}

The retained claimed rewards are inserted back into storage.

\begin{verbatim}
<Ledger<T>>::insert(&controller, &ledger);
\end{verbatim}

The Runtime is actually optimized to some degree: it only fetches a list of the highest staked nominators,
a maximum of 64. The rest gets no reward.

\begin{verbatim}
let exposure = <ErasStakersClipped<T>>::get(&era, &ledger.stash);
\end{verbatim}

Next, the function gets the era reward points from storage.

\begin{verbatim}
let era_reward_points = <ErasRewardPoints<T>>::get(&era);
\end{verbatim}

After that, the payout is split among the validator and its nominators. The validators receives the payment
first, creating an insertion into storage and sending a deposit event to the scheduler.

\begin{verbatim}
if let Some(imbalance) = Self::make_payout(
  &ledger.stash,
  validator_staking_payout + validator_commission_payout
) {
  Self::deposit_event(RawEvent::Reward(ledger.stash, imbalance.peek()));
}
\end{verbatim}

Then, the nominators receive a payout. The functions loops through the nominator list, conducting
a insertion into storage and a creation of a deposit event for each of the nominators.

\begin{verbatim}
for nominator in exposure.others.iter() {
  let nominator_exposure_part = Perbill::from_rational_approximation(
    nominator.value,
    exposure.total,
  );

  let nominator_reward: BalanceOf<T> = nominator_exposure_part * validator_leftover_payout;
  // We can now make nominator payout:
  if let Some(imbalance) = Self::make_payout(&nominator.who, nominator_reward) {
    Self::deposit_event(RawEvent::Reward(nominator.who.clone(), imbalance.peek()));
  }
}
\end{verbatim}

\subsubsection*{Considerations}

Based on the considerations described in section \ref{sect:primitive-types-considerations},
the analysis reveals multiple, critical points. Those must be covered in the benchmarking
processes described in section \ref{sect:benchmarking} and require preliminary work as
described in section \ref{sect:examples-preliminary-work}.
\newline

Key points:

\begin{itemize}
  \item The Ledger contains a varying list of claimed rewards. Fetching, retaining and searching through it can affect execution
  time. The retained list is inserted back into storage.
  \item The function searches the database for the reward points of the specified validator. If there are a lot of validators
  in the network, the search becomes more expensive.
  \item Looping through a list of nominators and creating IO operations for each heavily increases execution time.
  The Runtime fetches up to 64 nominators.
\end{itemize}

\subsection{Preliminary Work}\label{sect:examples-preliminary-work}
In order for certain benchmarks to produce conditions where resource heavy computation or excessive
I/O can be observed, the benchmarks might require some preliminary work on the environment, since those
conditions cannot be created with simply selected parameters.
As practical examples, this section describes the specifically designed benchmarks for the \verb|transfer|
and \verb|withdraw_unbonded| functions available in the Polkadot Runtime.

\subsubsection{Practical example \#1}
The $transfer$ function of the \textit{balances} module is designed to move the specified balance by the sender to the receiver.
The benchmark is configured to measure the function's worst possible condition:

\begin{itemize}
  \item Transfer will kill the sender account (by completely depleting the balance to zero).
  \item Transfer will create the recipient account (the recipient account doesn't have a balance yet).
\end{itemize}

\subsubsection*{Parameters}
The following parameters are selected:

\begin{center}
  \begin{tabular}{ l|r l l l }
    \textbf{Type} && \textbf{From} & \textbf{To} & \textbf{Description}\\
    \hline
    Account index & \verb|index| in... & 1 & 1000 & Used as a seed for account creation \\
    Balance & \verb|balance| in... & 2 & 1000 & Sender balance and transfer amount \\
  \end{tabular}
\end{center}

Executing a benchmark for each balance increment within the balance range for each index increment
within the index range will generate too many variants ($1000 \times 999$) and highly increase
execution time. Therefore, this benchmark is configured to first set the balance at value 1'000
and then to iterate from 1 to 1'000 for the index value. Once the index value reaches 1'000, the
balance value will reset to 2 and iterate to 1'000 (see algorithm \ref{sec:algo-benchmark-transfer}
for more detail):

\begin{itemize}
  \item \verb|index|: 1, \verb|balance|: 1000
  \item \verb|index|: 2, \verb|balance|: 1000
  \item \verb|index|: 3, \verb|balance|: 1000
  \item ...
  \item \verb|index|: 1000, \verb|balance|: 1000
  \item \verb|index|: 1000, \verb|balance|: 2
  \item \verb|index|: 1000, \verb|balance|: 3
  \item \verb|index|: 1000, \verb|balance|: 4
  \item ...
\end{itemize}

The parameters itself do not influence or trigger the two worst conditions and must be handled by
the implemented benchmarking tool. The $transfer$ benchmark is implemented as defined in algorithm
\ref{sec:algo-benchmark-transfer}.

\subsubsection*{Implementation}
The benchmarking implementation for the Polkadot Runtime function $transfer$ is defined as
follows (starting with the \textsc{Main} function):
\newline

\SetKw{KwInit}{Init:}
\SetKw{KwBy}{increment by}
\SetKwProg{Fn}{Function}{ is}{end}

\begin{algorithm}[H]\label{sec:algo-benchmark-transfer}
  \caption{Run multiple benchmark iterations for $transfer$ Runtime function}
  \SetAlgoLined
  \KwResult{$collection$: a collection of time measurements of all benchmark iterations}
  \BlankLine
  \Fn{\textsc{Main}}{
    \KwInit{collection = \{\}\;}
    \KwInit{$balance = 1'000$\;}
    \For{$index\gets1$ \KwTo $1'000$ \KwBy $1$}{
      $time \leftarrow$ \textsc{Run-Benchmark($index$, $balance$)}\;
      \textsc{Add-To($collection$, $time$)}\;
    }
    \BlankLine
    \KwInit{$index = 1'000$\;}
    \For{$balance\gets2$ \KwTo $1'000$ \KwBy $1$}{
      $time \leftarrow$ \textsc{Run-Benchmark($index$, $balance$)}\;
      \textsc{Add-To($collection$, $time$)}\;
    }
  }
  \BlankLine
  \Fn{\textsc{Run-Benchmark($index$, $balance$)}}{
    $sender \leftarrow$ \textsc{Create-Account(\textit{"caller"}, $index$)}\;
    $recipient \leftarrow$ \textsc{Create-Account(\textit{"recipient"}, $index$)}\;
    \textsc{Set-Balance($sender$, $balance$)}\;
    \BlankLine
    $time \leftarrow$\textsc{Timer(Transfer($sender$, $recipient$, $balance$))}\;
    \Return $time$
  }
\end{algorithm}

\begin{itemize}
  \item \textsc{Create-Account($name$, $index$)}
      \SubItem{Creates a Blake2 hash of the concatenated input of $name$ and $index$ representing
      the address of a account. This function only creates an address and does not conduct any I/O.}
  \item \textsc{Set-Balance($account$, $balance$)}
      \SubItem{Sets a initial $balance$ for the specified $account$ in the storage state.}
  \item \textsc{Transfer($sender$, $recipient$, $balance$)}
      \SubItem{Transfers the specified $balance$ from $sender$ to $recipient$ by calling the
      corresponding Runtime function. This represents the target Runtime function to be benchmarked.}
  \item \textsc{Add-To($collection$, $time$)}
      \SubItem{Adds a returned time measurement ($time$) to $collection$.}
  \item \textsc{Timer($function$)}
      \SubItem{Measures the time from the start of the specified $function$ to its completion.}
\end{itemize}

\subsubsection{Practical example \#2}
The $withdraw\_unbonded$ function of the \textit{staking} module is designed to move any unlocked
funds from the staking management system to be ready for transfer. The benchmark requires a couple
of I/O operations:

\begin{itemize}
  \item Create stash account and set initial balance.
  \item Create controller account and set initial balance.
  \item Bond a certain amount of the funds.
  \item Unbond full amount of the funds.
  \item Withdraw unbonded amount, making it ready for transfer.
\end{itemize}

\subsubsection*{Parameters}
The following parameters are selected:

\begin{center}
  \begin{tabular}{ l|r l l l }
    \textbf{Type} && \textbf{From} & \textbf{To} & \textbf{Description}\\
    \hline
    Account index & \verb|index| in... & 0 & 1000 & Used as a seed for account creation \\
  \end{tabular}
\end{center}

This benchmark does not require complex parameters. The values is use solely for account generation.

\subsubsection*{Implementation}
The benchmarking implementation for the Polkadot Runtime function $withdraw\_unbonded$ is defined as follows:
\newline

\SetKw{KwInit}{Init:}
\SetKw{KwBy}{increment by}
\SetKwProg{Fn}{Function}{ is}{end}

\begin{algorithm}[H]\label{sec:algo-benchmark-transfer}
  \caption{Run multiple benchmark iterations for $transfer$ Runtime function}
  \SetAlgoLined
  \KwResult{$collection$: a collection of time measurements of all benchmark iterations}
  \BlankLine
  \Fn{\textsc{Main}}{
    \KwInit{collection = \{\}\;}
    \For{$index\gets0$ \KwTo $1'000$ \KwBy $1$}{
      $stash \leftarrow$ \textsc{Create-Account(\textit{"stash"}, $index$)}\;
      $controller \leftarrow$ \textsc{Create-Account(\textit{"controller"}, $index$)}\;
      \textsc{Set-Balance($stash$, 100)}\;
      \textsc{Set-Balance($controller$, 100)}\;
      \textsc{Bond($stash$, $controller$, 10)}\;
      \textsc{UnBond($controller$, 10)}\;
      $time \leftarrow$\textsc{Timer(Withdraw-Unbonded($controller$))}\;
      \textsc{Add-To($collection$, $time$)}\;
    }
  }
  \BlankLine
\end{algorithm}

\begin{itemize}
  \item \textsc{Create-Account($name$, $index$)}
    \SubItem{Creates a Blake2 hash of the concatenated input of $name$ and $index$ representing
    the address of a account. This function only creates an address and does not conduct any I/O.}
  \item \textsc{Set-Balance($account$, $balance$)}
      \SubItem{Sets a initial $balance$ for the specified $account$ in the storage state.}
  \item \textsc{Bond($stash$, $controller$, $amount$)}
    \SubItem{Bonds the specified $amount$ for the $stash$ and $controller$ pair.}
  \item \textsc{UnBond($account$, $amount$)}
    \SubItem{Unbonds the specified $amount$ for the given $account$.}
  \item \textsc{Withdraw-Unbonded($controller$)}
    \SubItem{Withdraws the the full unbonded amount of the specified $controller$ account.
    This represents the target Runtime function to be benchmarked}
  \item \textsc{Add-To($collection$, $time$)}
    \SubItem{Adds a returned time measurement ($time$) to $collection$.}
  \item \textsc{Timer($function$)}
    \SubItem{Measures the time from the start of the specified $function$ to its completion.}
\end{itemize}

\section{Fees}
Block producers charge a fee in order to be economically sustainable. That fee must always
be covered by the sender of the transaction. Polkadot has a flexible mechanism to determine
the minimum cost to include transactions in a block.

\subsection{Fee Calculation}
Polkadot fees consists of three parts:

\begin{itemize}
\item Base fee: a fixed fee that is applied to every transaction and set by the Runtime.
\item Length fee: a fee that gets multiplied by the length of the transaction, in bytes.
\item Weight fee: a fee for each, varying Runtime function. Runtime implementers need to
      implement a conversion mechanism which determines the corresponding currency amount
      for the calculated weight.
\end{itemize}

The final fee can be summarized as:
\begin{eqnarray*}
\lefteqn{fee = base\ fee}\\
      &&{} + length\ of\ transaction\ in\ bytes \times length\ fee\\
      &&{} + weight\ to\ fee\\
\end{eqnarray*}

\subsection{Definitions in Polkadot}
The Polkadot Runtime defines the following values:
\begin{itemize}
\item Base fee: 100 uDOTs
\item Length fee: 0.1 uDOTs
\item Weight to fee conversion:
      $$
            weight\ fee = weight \times (100\ uDOTs \div (10 \times 10'000))
      $$
      A weight of 10'000 (the smallest non-zero weight) is mapped to $\frac{1}{10}$ of 100 uDOT.
      \newline
      This fee will never exceed the max size of an unsigned 128 bit integer.
\end{itemize}

\subsection{Fee Multiplier}
Polkadot can add a additional fee to transactions if the network becomes too busy and starts to
decelerate the system. This fees can create incentive to avoid the production of
low priority or insignificant transactions. In contrast, those additional fees will decrease if
the network calms down and it can execute transactions without much difficulties.
\newline

That additional fee is known as the \verb|Fee Multiplier| and its value is defined
by the Polkadot Runtime. The multiplier works by comparing the saturation of blocks; if the previous 
block is less saturated than the current block (implying an uptrend), the fee is slightly increased.
Similarly, if the previous block is more saturated than the current block (implying a downtrend), the
fee is slightly decreased.
\newline

The final fee is calculated as:
$$
      final\ fee = fee \times Fee\ Multiplier
$$

\subsubsection{Update Multiplier}
The \verb|Update Multiplier| defines how the multiplier can change. The Polkadot Runtime internally
updates the multiplier after each block according the following formula:

\begin{eqnarray*}
diff &=& (target\ weight - previous\ block\ weight)\\
v &=& 0.00004\\
next\ weight &=& weight \times (1 + (v \times diff) + (v \times diff)^2 / 2)\\
\end{eqnarray*}

Polkadot defines the \verb|target_weight| as 0.25 (25\%). More information about this algorithm is described
in the Web3 Foundation research paper: \url{https://research.web3.foundation/en/latest/polkadot/Token%20Economics.html#relay-chain-transaction-fees-and-per-block-transaction-limits}.

\end{document}
