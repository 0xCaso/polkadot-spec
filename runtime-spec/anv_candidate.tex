\subsection{Candidate Selection}
\label{sect-primary-validation}

Collators produce candidates as defined in Definition \ref{defn-candidate}.
Validators verify the validity of the received candidates as described in
Algorithm \ref{algo-primary-validation}. \todo{@fabio: confirm this} Validators
back the validity respectively the invalidity by extending the candidates into 
candidate receipts as defined in Definition \ref{defn-candidate-receipt} and
communicate it by issuing statements as defined in Definition
\ref{defn-gossip-statement}.
\newline

The validator ensures the that every candidate considered for inclusion has at least
one other validator backing it. Candidates without backing are discarded. Backed candidates 
which later prove to be invalid qualify the backer for slashing.
\newline

It's note worthy that validators do not dictate which candidates are prioritized
for inclusion. Each validator decides for itself - on whatever metric - which
\textit{valid} candidates to include.

\begin{definition}
  \label{defn-candidate}
  A \textbf{candidate}, $C_{coll}(PoV_B)$, is issues by collators and contains the PoV
  block and enough data in order for any validator to verify its validity. A
  candidate is a tuple of the following format:
  \[
  C_{coll}(PoV_B) := (id_p, h_b({B_{^{relay}_{parent}}}), id_{C}, Sig^{Collator}_{SR25519}, Head_p(B), h_b({PoV_B}))
  \]

  where each value represents:
  \begin{itemize}
    \item $id_p$: the Parachain Id this candidate is for.
    \item $h_b({B_{^{relay}_{parent}}})$: the hash of the relay chain block that this
    candidate should be executed in the context of.
    \item $id_C$: the Collator relay-chain account ID as defined in Definition
    \todo{@fabio}.
    \item $Sig^{Collator}_{SR25519}$: the signature on the 256-bit Blake2 hash
    of the block data by the collator.
    \item $Head_p(B)$: the head data of parachain block $B$ as defined in
    Definition \ref{defn-head-data}.
    \item $h_b({PoV_B})$: the hash of the PoV block.
    \todo{@fabio}.
  \end{itemize}

\end{definition}

\begin{definition}
  \label{defn-head-data}
  The \textbf{head data}, $Head_p(B)$, of a parachain block is a tuple of the following format:
  \[
    (H_i(B), H_p(B), H_r(B))
  \]

  Where $H_i(B)$ is the block number of parachain block $B$, $H_p(B)$ is the
  32-byte Blake2 hash of the parent block header and $H_r(B)$ represents the
  root of the post-execution state.
  \todo{@fabio: clarify if $H_p$ is the hash of the header or full block}
  \todo{@fabio: maybe define those symbols at the start (already defined in the Host spec)?}
\end{definition}

\begin{algorithm}[H]
  \caption[]{\sc PrimaryValidation}
  \label{algo-primary-validation}
  \begin{algorithmic}[1]
    \Require{$B$, $\pi_B$, relay chain parent block $B^{relay}_{parent}$}
    %%  \Ensure{}

    \State Retrieve $v_B$ from the relay chain state at $B^{relay}_{parent}$
    \State Run Algorithm \ref{algo-validate-block} using $B, \pi_B, v_B$
  \end{algorithmic}
\end{algorithm}

\begin{algorithm}[H]
  \caption[]{\sc ValidateBlock}
  \label{algo-validate-block}
  \begin{algorithmic}[1]
    \Require{$B, \pi_B, v_B$}
    %%  \Ensure{}
    \State retrieve the runtime code $R_\rho$ that is specified by $v_B$ from the relay chain state.
    \State check that the initial state root in $\pi_B$ is the one claimed in $v_B$
    \State Execute $R_\rho$ on $B$ using $\pi_B$ to simulate the state.
    \State If the execution fails, return fail.
    \State Else return success, the new header data $h_B$ and the outgoing messages $M$. \todo{@fabio: same as head data?}
  \end{algorithmic}
\end{algorithm}
