\documentclass{article}
\usepackage[english]{babel}

%%%%%%%%%% Start TeXmacs macros
\newcommand{\tmem}[1]{{\em #1\/}}
\newcommand{\tmsamp}[1]{\textsf{#1}}
\newcommand{\tmstrong}[1]{\textbf{#1}}
\newcommand{\tmtextbf}[1]{{\bfseries{#1}}}
\newcommand{\tmtextit}[1]{{\itshape{#1}}}
%%%%%%%%%% End TeXmacs macros

\begin{document}

\title{Polkadot RE}

\date{September 18, 2018}

\maketitle

\section{Block}

In Polkadot RE, a block is made of two main parts, namely the \tmtextit{block
header} and the \tmtextit{list of extrinsics}. {\tmem{The Extrinsics}}
represent the generalization of the concept of {\tmem{transaction}},
containing any set of data that is external to the system and which the
underlying chain wishes to validate and keep track of.

\subsection{Block Header}

The block header is designed to be minimalistic in order to boost the
efficiency of the light clients. It contains the following elements:

\begin{itemize}

\item \tmtextbf{{\tmsamp{parent\_hash:}}} is the 32-byte Blake2s hash of the header
of the parent of the block, indicated hence forth by {\tmtextbf{$H_p$}}.

\item {\tmstrong{{\tmsamp{number:}}}} formally indicated as {\tmstrong{$H_i$}} is an
integer, which represents the index of the current block in the chain. It is
equal to the number of the ancsestor blocks. The genesis block has number 0.

\item {\tmstrong{{\tmsamp{state\_root:}}}} formally indicated as {\tmstrong{$H_r$}}
is the root of the Merkle trie whose leaves implements the storage for the
system.

\item {\tmstrong{{\tmsamp{extrinsics\_root:}}}} is the root of the Merkle trie whose
leaves represent individual extrinsic being validated in this block. This
element is formally referred to as {\tmstrong{$H_e$}}.

\item {\tmstrong{{\tmsamp{digest:}}}} this field is used to store any chain-specific
auxiliary data, which could help the light clients interact with the block
without the need of accessing the full storage. Polkadot RE does not impose any limitation or specification for this field. It essentially can be a byte array of any length. This field is indicated as
{\tmstrong{$H_d$}}
\end{itemize}

\section{Entry into Runtime}

\section{API}

\

\end{document}
