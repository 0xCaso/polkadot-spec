\documentclass{article}
\usepackage[english]{babel}
\usepackage{amsmath,amssymb}

%%%%%%%%%% Start TeXmacs macros
\newcommand{\assign}{:=}
\newcommand{\cdummy}{\cdot}
\newcommand{\nobracket}{}
\newcommand{\nosymbol}{}
\newcommand{\tmem}[1]{{\em #1\/}}
\newcommand{\tmname}[1]{\textsc{#1}}
\newcommand{\tmop}[1]{\ensuremath{\operatorname{#1}}}
\newcommand{\tmrsub}[1]{\ensuremath{_{\textrm{#1}}}}
\newcommand{\tmsamp}[1]{\textsf{#1}}
\newcommand{\tmstrong}[1]{\textbf{#1}}
\newcommand{\tmtextbf}[1]{{\bfseries{#1}}}
\newcommand{\tmtextit}[1]{{\itshape{#1}}}
\newcommand{\tmverbatim}[1]{{\ttfamily{#1}}}
\newtheorem{definition}{Definition}
\newtheorem{notation}{Notation}
%%%%%%%%%% End TeXmacs macros

\begin{document}

\title{Polkadot RE Spec}

\date{November 21, 2018}

\maketitle

\section{Conventions and Definitions}

\begin{definition}
  {\tmstrong{Runtime}} is the state transition function of the decentralized
  ledger protocol.
\end{definition}

\begin{definition}
  \label{def-path-graph}A {\tmstrong{path graph}} or a {\tmstrong{path}} of
  $n$ nodes, formally referred to as {\tmstrong{$P_n$}}, is a tree a with two
  nodes of vertex degree 1 and the other n-2 nodes of vertex degree 2.
  Therefore, $P_n$can be represented by sequences of $(v_1, \ldots, v_n)$
  where $e_i = (v_i, v_{i + 1})$ for $1 \leqslant i \leqslant n - 1$ is the
  edge which connect $v_i$ and $v_{i + 1}$.
\end{definition}

\begin{definition}
  \label{def-radix-tree}{\tmstrong{radix r tree}} is a variant of \ a trie in
  which:
  \begin{itemize}
    \item Every node has at most $r$ children where $r = 2^x$ for some $x$.
    
    \item Each node that is the only child of a parent which does not
    represent a valid key is merged with its parent.
  \end{itemize}
\end{definition}

As the result, in a radix tree, any path whose interior vertices all have only
one child and does not represent a valid key in the data set, is compressed
into a single edge. This improves space efficiency when the key space is
sparse.

\begin{definition}
  The by a {\tmstrong{sequences of bytes}} or a {\tmstrong{byte array}}, $b$,
  of length $n$, we refer to
  \[ b \assign (b_0, b_1, ..., b_{n - 1})  \text{such that } 0 \leqslant b_i
     \leqslant 255 \]
  We define $\mathbb{B}_n$ to be the {\tmstrong{set of all byte arrays of
  length $n$}}. Furthermore, we define:
  \[ \mathbb{B} \assign \bigcup^{\infty}_{i = 0} \mathbb{B}_i \]
\end{definition}

\begin{notation}
  We represent the concatination of byte arrays $a \assign (a_0, \ldots, a_n)$
  and $b \assign (b_0, \ldots, b_m)$ by:
  \[ a || b \assign (a_0, \ldots, a_n, b_0, \ldots, b_m) \]
\end{notation}

\begin{definition}
  \label{defn-bit-rep}For a given byte $b$ the {\tmstrong{bitwise
  representation}} of $b$ is defined as
  \[ b \assign b^7 \ldots b^0 \]
  where
  \[ b = 2^0 b^0 + 2^1 b^1 + \cdots + 2^7 b^7 \]
\end{definition}

\section{Block}

In Polkadot RE, a block is made of two main parts, namely the \tmtextit{block
header} and the \tmtextit{list of extrinsics}. {\tmem{The Extrinsics}}
represent the generalization of the concept of {\tmem{transaction}},
containing any set of data that is external to the system, and which the
underlying chain wishes to validate and keep track of.

\subsection{Block Header}\label{block}

The block header is designed to be minimalistic in order to boost the
efficiency of the light clients. It is defined formally as follows:

\begin{definition}
  \label{def-block-header}The header of block B, $\tmop{Head} (B)$ is a
  5-tuple containing the following elements:
  \begin{itemize}
    \item \tmtextbf{{\tmsamp{parent\_hash:}}} is the 32-byte Blake2s hash of
    the header of the parent of the block indicated henceforth by
    \tmtextbf{$H_p$}.
    
    \item {\tmstrong{{\tmsamp{number:}}}} formally indicated as
    {\tmstrong{$H_i$}} is an integer, which represents the index of the
    current block in the chain. It is equal to the number of the ancestor
    blocks. The genesis block has number 0.
    
    \item {\tmstrong{{\tmsamp{state\_root:}}}} formally indicated as
    {\tmstrong{$H_r$}} is the root of the Merkle trie, whose leaves implement
    the storage for the system.
    
    \item {\tmstrong{{\tmsamp{extrinsics\_root:}}}} is the root of the Merkle
    trie, whose leaves represent individual extrinsic being validated in this
    block. This element is formally referred to as {\tmstrong{$H_e$}}.
    
    \item {\tmstrong{{\tmsamp{digest:}}}} this field is used to store any
    chain-specific auxiliary data, which could help the light clients interact
    with the block without the need of accessing the full storage. Polkadot RE
    does not impose any limitation or specification for this field. It
    essentially can be a byte array of any length. This field is indicated as
    {\tmstrong{$H_d$}}
  \end{itemize}
\end{definition}

\subsection{Justified Block Header}

The Justified Block Header is provided by the consensus engine and presented
to the Polkadot RE, in order for the block to be appended to the blockchain.
It contains the following parts:
\begin{itemize}
  \item {\tmstrong{{\tmsamp{{\tmstrong{block\_header}}}}}} the complete block
  header as defined in Section \ref{block} and denoted by $\tmop{Head} (B)$.
  
  \item {\tmstrong{{\tmsamp{justification}}}}: as defined by the consensus
  specification denoted by $\tmop{Just} (B)$ {\todo{link this to its
  definition from consensus}}.
  
  \item {\tmstrong{{\tmsamp{authority Ids}}}}: This is the list of the Ids of
  authorities, which have voted for the block to be stored and is formally
  referred to as $A (B)$. An authority Id is 32bit.
\end{itemize}

\subsection{Extrinsics}

Each block also contains a list of extrinsics. Polkadot RE does not specify or
limit the internal of each extrinsics beside the fact that each extrinsics is
a blob of encoded data. The {\tmsamp{extrinsics\_root}} should correspond to
the root of the Merkle trie, whose leaves are made of the block extrinsics
list.

\section{Entry into Runtime}

\section{API}

\subsection{Block Submission and Validation}

Block validation is the process, by which the client asserts that a block is
fit to be added to the blockchain. That is to say, the block is consistent
with the world state and transitions from the state of the system to a new
valid state.

\

Blocks can be handed to the Polkadot RE both from the network stack and from
consensus engine.

\

Both the runtime and the Polkadot RE need to work together to assure block
validity. This can be accomplished by Polkadot RE invoking
\tmverbatim{execute\_block} entry into the runtime as a part of the validation
process.

\

Polkadot RE implements the following procedure to assure the validity of the
block:

{\algorithm{{\tmname{Import-and-Validate-Block($B, \tmop{Just} (B)$)}}

1\quad{\tmname{Verify-Block-Justification}}$(B, \tmop{Just} (B))$

2\quad Verify $H_{p (B)} \in \tmop{Blockchain}$.

3\quad State-Changes = Runtime.{\tmname{Execute-Block$(B)$}}

4\quad{\tmname{Update-World-State}}(State-Changes)}}

\subsection{Storage Access}

\section{State Storage and the Storage Trie}

For storing the state of the system, Polkadot RE implements a hash table
storage where the keys are used to access each data entry state. There is no
limitation on neither the size of the key nor the size of the data stored
under them, beside the fact that thay are byte arrays.

To authenticate the state of the system, the stored data is re-arranged and
hashed in a {\tmem{radix 16 tree}} also known as {\tmem{base-16 modified
Merkle Patricia Tree}}, which hereafter we simply refer to as the
{\tmem{{\tmstrong{Trie}},}} in order to compute the hash of the whole state
storage consistently and efficiently at any given time.

Also modification has been done in storing the nodes' hash in the Merkle tree
structure to save space on entries storing small entries.

Because the Tri is used to compute the {\tmem{state root}}, $H_r$, (see
Definition \ref{def-block-header}), which is used to authenticate the validity
of the state database, Polkadot RE follows a rigorous encoding algorithm to
compute the values stored in the trie nodes to ensure that the computed Merkle
hash, $H_r$, matches across clients.

\subsection{The General Tree Structure}

As the trie is a radix 16 tree, in this sense, each key value identifies a
unique node in the tree. However, a node in a tree might or might not be
associated with a key in the storage.

To identify the node corresponding to a key value, $k$, first we need to
encode $k$ in a uniform way:

\begin{definition}
  The for the purpose labeling the branches of the Trie key $k$ is encoded to
  $k_{\tmop{enc}}$ using KeyEncode functions:
  \begin{equation}
    k_{\tmop{enc}} \assign (k_{\tmop{enc}_1}, \ldots, k_{\tmop{enc}_{2 n}})
    \assign \tmop{KeyEncode} (k) \label{key-encode-in-trie}
  \end{equation}
  such that:
  \[ \tmop{KeyEncode} (k) : \left\{ \begin{array}{lll}
       \mathbb{B}^{\nosymbol} & \rightarrow & \tmop{Nibbles}_4\\
       k \assign (b_1, \ldots, b_n) \assign & \mapsto & (b^1_1, b^2_1, b_2^1,
       b^2_2, \ldots, b^1_n, b^2_n)\\
       &  & \assign (k_{\tmop{enc}_1}, \ldots, k_{\tmop{enc}_{2 n}})
     \end{array} \right. \]
  where $\tmop{Nibble}_4$ is the set of all nibbles of 4-bit arrays and
  $b^1_i$ and $b^2_i$ are 4-bit nibbles which are the the little endian
  representation of $b_i$:
  \[ (b^1_i, b^2_i) \assign (b_1 \tmop{mod} 16, b_2 / 16) \]
  where mod is the remainder and / is the integer division operators.
\end{definition}

By looking at $k_{\tmop{enc}}$ as a sequence of nibbles, can walk the radix
tree to reach the node identifying the storage value of $k$.

\subsection{The Merkle proof}

To prove the consistency of the state storage across network and its
modifications efficiently, the Merkle hash of the storage trie need to be
computed regirously.

The Merkle hash of the trie is computed recursively. As such hash value of
each node depends on the hash value of all its children and also on its value.
Therefore it suffice to define how to compute the hash value of a typical node
as a function of hash value of its children and its own value.

\begin{definition}
  Suppose node N of storage state trie has key value $k_N$, and parent key
  value of $k_{P (N)}$, such that:
  \[ \tmop{KeyEncode} (k_N) = (k_{\tmop{enc}_1}, \ldots, k_{\tmop{enc}_{i -
     1}}, k_{\tmop{enc}_i}, \ldots, k_{\tmop{enc}_{2 n}}) \]
  and
  \[ \tmop{KeyEncode} (k_{P (N)}) = (k_{\tmop{enc}_1}, \ldots,
     k_{\tmop{enc}_{i - 1}}) \]
  We define
  \[ \tmop{pk}_N \assign (k_{\tmop{enc}_i}, \ldots, k_{\tmop{enc}_{2 n}}) \]
  to be the {\tmstrong{the partial key}} of N.
\end{definition}

\begin{definition}
  \label{def-node-prefix}For a trie node N, {\tmstrong{Node Prefix }}function
  is a value specifying the node type as follows:
  \[ \tmop{NodePrefix} (N) \assign \left\{ \begin{array}{lll}
       1 &  & N \text{is a leaf node}\\
       254 &  & N \text{is a branch node without value}\\
       255 &  & N \text{is a branch node with value}
     \end{array} \right. \]
\end{definition}

\begin{definition}
  For a given node $N$, with partial key of $\tmop{pk}_N$ and Value $v$, the
  {\tmstrong{encoded representation}} of $N$, formally referred to as
  $\tmop{Enc}_{\tmop{Node}} (N)$ is determined as fallows, in case which:
  \begin{itemize}
    \item $N$ is a leaf node:
    \[ \tmop{Enc}_{\tmop{Node}} (N) \assign \tmop{Enc}_{\tmop{len}} (N) ||
       \tmop{HPE} (\tmop{pk}_N) || \tmop{Enc}_{\tmop{PC}} (v) \]
    \item N is a branch node:
    \[ \begin{array}{l}
         \text{Enc\tmrsub{Node}\left(N\right)} \assign\\
         \nobracket \tmop{NodePrefix} (N) || \tmop{ChildrenBitmap} (N) \|
         \tmop{HPE}_{\tmop{PC}} (v) || \tmop{Enc}_{\tmop{PC}}
         (\tmop{Enc}_{\tmop{Node}}) ||\\
         \tmop{Enc}_{\tmop{PC}} (N_{C_1}) \ldots \tmop{Enc}_{\tmop{PC}}
         (N_{C_n})
       \end{array} \]
  \end{itemize}
\end{definition}

Where $N_{C_1} \ldots N_{C_n}$ with $n \leqslant 16$ are the children nodes of
$N$.

\begin{definition}
  For a given node $N$, the {\tmstrong{Merkle value}} of $N$, denoted by $H
  (N)$ is defined as follows:
  \[ \begin{array}{lll}
       & H : \mathbb{B} \rightarrow \bigcup_{i = 0^{\nosymbol}}^{32}
       \mathbb{B}_i & \\
       & H (N) : \left\{ \begin{array}{lll}
         \tmop{Enc}_{\tmop{Node}} (N) & \| \tmop{Enc}_{\tmop{Node}} (N)\|< 32
         & \\
         \tmop{Hash} (\tmop{Enc}_{\tmop{Node}} (N)) & \|
         \tmop{Enc}_{\tmop{Node}} (N)\| \geqslant 32 & 
       \end{array} \right. & 
     \end{array} \]
\end{definition}

\section{Extrinsics trie}

Although the same radix-16 tree is being used to verify the consistency of the
Extrinsics data, unlike the Storage trie, Polkadot RE {\tmem{does not}} use a
Merkle tree structure to accomplish this task. Instead, it stores the entire
trie in a byte array structure and finally computes its hash as a byte array,
as defined in Definition \ref{def-closed-form-trie}.

\

\begin{definition}
  \label{def-closed-form-trie}{\tmstrong{Closed Form Trie}} {\todo{TODO:
  Define according to substerate-submods/trie/trie-root/src/lib.rs}}
\end{definition}

\section{Auxilary Encodings}

\subsection{Parity Codec}

Polkadot RE uses Parity Codec to encode byte arrays to provide canonical
encoding and to produce consistent hash values across their implementation,
including the Merkle hash proof for the State Storage.

\

\begin{definition}
  The parity codec of a byte array $A$:
  \[ A \assign b_1 b_2 \ldots b_n \]
  such that $n < 2^{536}$ is a byte array refered to $\tmop{Enc}_{\tmop{PC}}
  (A)$ and defined as follows:
  \[ \tmop{Enc}_{\tmop{PC}} (A) \assign \left\{ \begin{array}{lll}
       l^{\nosymbol}_1 b_1 b_2 \ldots b_n &  & 0 \leqslant n < 2^6\\
       i^{\nosymbol}_1 i^{\nosymbol}_2 b_1 \ldots b_n &  & 2^6 \leqslant n <
       2^{14}\\
       j^{\nosymbol}_1 j^{\nosymbol}_2 j_3 b_1 \ldots b_n &  & 2^{14}
       \leqslant n < 2^{30}\\
       k_1^{\nosymbol} k_2^{\nosymbol} \ldots k_m^{\nosymbol} b_1 \ldots b_n &
       & 2^{14} \leqslant n
     \end{array} \right. \]
  in which:
  \[ \begin{array}{|l|}
       \hline
       l^1_1 l_1^0 = 00\\
       \hline
       i^1_1 i_1^0 = 01\\
       \hline
       j^1_1 j_1^0 = 10\\
       \hline
       k^1_1 k_1^0 = 11\\
       \hline
     \end{array} \]
  and n is stored in $\tmop{Enc}_{\tmop{PC}} (A)$ in little-endian format in
  base-2 as follows:
  \[ n = \left\{ \begin{array}{lll}
       l^7_1 \ldots l^3_1 l^2_1 &  & n < 2^6\\
       i_2^7 \ldots i_2^0 i_1^7 \ldots i^2_1^{\nosymbol} &  & 2^6 \leqslant n
       < 2^{14}\\
       j_4^7 \ldots j_4^0 j_3^7 \ldots j_1^7 \ldots j^2_1 &  & 2^{14}
       \leqslant n < 2^{30}\\
       k_2 + k_3 2^8 + k_4 2^{2 \cdummy 8} + \cdots + k_m 2^{(m - 2) 8} &  &
       2^{30} \leqslant n
     \end{array} \right. \]
  where:
  \[ m = l^7_1 \ldots l^3_1 l^2_1 + 4 \]
\end{definition}

\subsection{Hex Encoding}

Practically it is more convient and efficient to store and process data which
is stored in a byte arrray. On the other hand, radix-16 tree keys are broken
in 4-bits nibbles. Accordingly, we need a method to canonically encode
sequences of 4-bits nibbles into byte arrays:

\

\begin{definition}
  \label{def-hpe}Suppose that $\tmop{PK} = (k_1, \ldots, k_n)$ is a sequence
  of nibbles, then
  
  \begin{tabular}{l}
    $\tmop{Enc}_{\tmop{HE}} (\tmop{PK}) \assign$\\
    $\left\{ \begin{array}{lll}
      \tmop{Nibbles}_4 & \rightarrow & \mathbb{B}\\
      \tmop{PK} = (k_1, \ldots, k_n) & \mapsto & \left\{ \begin{array}{l}
        \begin{array}{ll}
          (0, k_1 + 16 k_2, \ldots, k_{2 i - 1} + 16 k_{2 i}) & n = 2 i\\
          (k_1, k_2 + 16 k_3, \ldots, k_{2 i} + 16 k_{2 i + 1}) & n = 2 i + 1
        \end{array}
      \end{array} \right.
    \end{array} \right.$
  \end{tabular}
\end{definition}

\subsection{Partial Key Encoding}

\begin{definition}
  \label{def-key-len-enc}Let $N$ be a node in the storage state trie with
  Partial Key $\tmop{PK}_N$. We define the {\tmstrong{Partial key length
  encoding}} function formally referred to as $\tmop{Enc}_{\tmop{len}} (N)$ as
  follows:
  \[ \begin{array}{ll}
       \tmop{Enc}_{\tmop{len}} (N) & \assign\\
       \tmop{NodePrefix} (N) & +\\
       \left\{ \begin{array}{lllll}
         (\| \tmop{PK}_N \|) &  & \tmop{NisleafNode} & \& & \| \tmop{PK}_N \|<
         127\\
         (127) || (\tmop{LE} (\| \tmop{PK}_N \|- 127)) &  &
         \tmop{NisaleafNode} & \& & \| \tmop{PK}_N \| \geqslant 127
       \end{array} \right. & 
     \end{array} \]
  where $\tmop{NodePrefix}$ function is defined in Definition
  \ref{def-node-prefix}.
  
  \ 
\end{definition}

\end{document}
